\vspace{-2pt}
\section{Related work\label{se:re}}
Optimizations on computation, communication, throughput and energy consumption are always conflicting~\cite{lin2012communication,IEICE14,FM15}. When facing multiple objectives, a solution is to formulate objectives with different weights~\cite{IEICE14}, thus transforming multi-objective into one to be solved with integer linear solvers.
However, the shortage of scalability limits its application for realistic problems.
The alternative methods, such as stepwised optimization~\cite{IEICE14}, the integration of evolutionary algorithms~\cite{lin2012communication}, may still fall into the problem of local optimization, and the results may not be real Pareto fronts.

Search based algorithms are also considered in the literature for mapping and scheduling optimization of MPSoCs. Zhu et. al. model the constraints and task dependency into timed automata and compute Pareto front with model checking techniques~\cite{FM15}.
\cite{legriel2010approximating} provides a multi-dimensional binary search algorithm for approximating the Pareto fronts and guaranteeing the computable bounds. These techniques also suffer from scalability problems.

To obtain Pareto optimal for large scale systems, \cite{thiele2007mapping} addresses the mapping problems with multiple optimization objectives, based on evolutionary algorithms. \cite{yassa2013genetic} applies genetic algorithms for multi-objective optimization in workflow scheduling. Among the techniques, the Non-dominated Sorting Genetic Algorithm (NSGAII)~\cite{deb2002fast} is recognized as one of the most classical framework for multi-objective optimization problems. %Addition to these methods,
%\cite{tetzlaffintelligent} applies machine learning technique to learn the program behaviour and provides communication and power-efficient mapping strategies.
In this setting, initial solution and heuristics are important factors for the efficiency of the algorithms. On the other hand, local search based methods have achieved great success in solving various combinatorial problems by obtaining near-optimal solutions with a small cost~\cite{wang2016two}. Therefore, the aim of this paper is to integrate a local search technique into a genetic framework to improve both the efficiency and accuracy when computing Pareto optimals.

%\textbf{Real-world issues are often incomputable by exact methods which waste much time. Then local search approaches are introduced accordingly and proved to be significant for combinatorial optimization problems. Local search techniques~\cite{Ishibuchi98} are always applied to solve flowshop scheduling problems.  \cite{lourencco2003iterated} firstly proposes an iterated local search in which a perturbation is applied to the solution generated by the last search phase. Subsequently, \cite{hansen2003variable} developed a time-saving local search called variable neighbourhood search, aming to reduce the searching space. In recent years, many researchers focus on extensions of the local search for multi-objective problems. \cite{moalic2013fast} proposed a fast local search approach for multi-objective problems based on pareto optimal. Although optimal solutions cannot be guaranteed by local search methods, near-optimal solutions can be obtained with reasonable time. This characteristic is of importance especially for large-scale problems.}
